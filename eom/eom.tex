\documentclass[a4paper,fleqn,12pt]{article}
\usepackage[T1]{fontenc}
%\usepackage[utf8]{inputenc}
\usepackage[brazilian]{babel}
\usepackage[left=2.5cm,right=2.5cm,top=3cm,bottom=2.5cm]{geometry}
\usepackage{mathtools}
%\usepackage{amsthm}
\usepackage{amsmath}
\usepackage{nccmath}
\usepackage{amssymb}
\usepackage{amsfonts}
\usepackage{physics}
\usepackage{dsfont}
%\usepackage{mathrsfs}

\usepackage{titling}
\usepackage{indentfirst}

\usepackage{bm}
\usepackage[dvipsnames]{xcolor}
\usepackage{cancel}

\usepackage{xurl}
\usepackage[colorlinks=true]{hyperref}

\usepackage{float}
\usepackage{graphicx}
\usepackage{tikz}
\usepackage{caption}
\usepackage{subcaption}

%%%%%%%%%%%%%%%%%%%%%%%%%%%%%%%%%%%%%%%%%%%%%%%%%%%

\newcommand{\eps}{\epsilon}
\newcommand{\vphi}{\varphi}
\newcommand{\cte}{\text{cte}}

\newcommand{\N}{\mathbb{N}}
\newcommand{\Z}{\mathbb{Z}}
\newcommand{\Q}{\mathbb{Q}}
\newcommand{\R}{\mathbb{R}}
\newcommand{\C}{\mathbb{C}}
\renewcommand{\H}{\hat{H}}
\newcommand{\intR}{\int_{-\infty}^{\infty}}

\newcommand{\0}{\vb{0}}
\newcommand{\1}{\mathds{1}}
\newcommand{\E}{\vb{E}}
\newcommand{\B}{\vb{B}}
\renewcommand{\v}{\vb{v}}
\renewcommand{\r}{\vb{r}}
\renewcommand{\k}{\vb{k}}
\newcommand{\p}{\vb{p}}
\newcommand{\q}{\vb{q}}
\newcommand{\F}{\vb{F}}

\renewcommand{\a}{\hat{a}}
\renewcommand{\b}{\hat{b}}
\renewcommand{\c}{\hat{c}}
\newcommand{\nn}{\hat{n}}

\newcommand{\gf}[2]{\ev{\ev{#1 : #2}}}
\newcommand{\zub}[2]{\ev{\comm{#1}{#2}_\mp}}

\newcommand{\s}[1]{\mathcal{#1}}
%\newcommand{\prodint}[2]{\left\langle #1 , #2 \right\rangle}
\newcommand{\cc}[1]{\overline{#1}}
\newcommand{\Eval}[3]{\eval{\left( #1 \right)}_{#2}^{#3}}

\newcommand{\unit}[1]{\; \mathrm{#1}}

\newcommand{\n}{\medskip}
\newcommand{\e}{\quad \mathrm{e} \quad}
\newcommand{\ou}{\quad \mathrm{ou} \quad}
\newcommand{\virg}{\, , \;}
\newcommand{\ptodo}{\forall \,}
\renewcommand{\implies}{\; \Rightarrow \;}
%\newcommand{\eqname}[1]{\tag*{#1}} % Tag equation with name

\setlength{\droptitle}{-8em}


\title{\Huge{\textbf{EOM}}}
\author{Mateus Marques}

\begin{document}

\maketitle

Tentarei justificar as aproximações EOM0 e EOM1 do paper ``Quantitative comparison of Anderson impurity solvers applied to transport in quantum dots''.

\n

Primeiramente, alguns fatos básicos:
\begin{itemize}
\item Equação de movimento na notação de Zubarev:
$$
\omega^+ \gf{A}{B} = \ev{\{ A, B\}} + \gf{[A, H]}{B}.
$$
\item Básicos:
$$
[a, bc] = [a, b] c + b [a, c] \e [a, bc] = \{a, b\} c - b \{a, c\}.
$$
$$
[f_\sigma, n_{\sigma}] = \{f_\sigma, n_{\sigma}\} = f_\sigma \e
[f_\sigma^\dagger, n_{\sigma}] = -\{f_\sigma^\dagger, n_{\sigma}\}
= - f_\sigma^\dagger.
$$
\end{itemize}


O modelo SIAM (Single Impurity Anderson Model) em consideração é
$$
H = H_{\text{imp}} + H_{\text{C}} + H_{\text{B}},
$$
onde
$$
H_{\text{imp}} = \sum_\sigma \eps_\sigma f_\sigma^\dagger f_\sigma +
U n_\uparrow n_\downarrow,
$$
$$
H_{\text{B}} = \sum_{\alpha, \k, \sigma} \eps_{\alpha \k \sigma}
c_{\alpha \k \sigma}^\dagger c_{\alpha \k \sigma},
$$
$$
H_{\text{C}} = \sum_{\alpha, \k, \sigma} \Big(
V_{\alpha \k \sigma} c^\dagger_{\alpha \k \sigma} f_\sigma +
V_{\alpha \k \sigma}^* f^\dagger_\sigma  c_{\alpha \k \sigma}
\Big).
$$

Temos que $\alpha \in \{L, R\}$ e assumimos que $\eps_{\alpha \k \sigma}$ e $V_{\alpha \k \sigma}$ não dependem de $\sigma$. A equação de movimento exata para $G_\sigma(\omega^+) = \gf{f_\sigma}{f_\sigma^\dagger}$ é a mesma que calculamos em nossos estudos anteriores:
$$
\Big(
\omega^+ - \eps_\sigma - \Sigma_0(\omega^+)
\Big) G_\sigma(\omega^+)
= 1 + U \gf{n_{\cc{\sigma}} f_\sigma}{f_\sigma^\dagger},
$$
onde $\displaystyle{\Sigma_0(\omega^+) = \sum_{\alpha \k} \frac{\abs{V_{\alpha \k}^2}}{\omega - \eps_{\alpha \k}}}$. Lembremos que, para chegar na equação acima, tínhamos calculado os comutadores
$$
[f_\sigma, H] = \eps_\sigma  f_\sigma + U n_{\cc{\sigma}} f_\sigma +
\sum_{\alpha \k} V_{\alpha \k}^* c_{\alpha \k \sigma},
$$
$$
[c_{\alpha \k \sigma}, H] = \eps_{\alpha \k} c_{\alpha \k \sigma} +
V_{\alpha \k} f_\sigma.
$$


\section{Aproximação EOM0}

A aproximação que descobri que dá o resultado correto para EOM0 (equação (11) do paper) é $[n_{\cc{\sigma}}, H] \approx 0$.

Vamos mais fundo calculando a equação de movimento para $\gf{n_{\cc{\sigma}} f_\sigma}{f_\sigma^\dagger}$:
$$
\omega^+ \gf{n_{\cc{\sigma}} f_\sigma}{f_\sigma^\dagger} =
\ev{\{n_{\cc{\sigma}} f_\sigma, f_\sigma^\dagger\}} +
\gf{[n_{\cc{\sigma}} f_\sigma, H]}{f_\sigma^\dagger}.
$$
Temos que
$$
\{n_{\cc{\sigma}} f_\sigma, f_\sigma^\dagger\} =
n_{\cc{\sigma}} \{f_\sigma, f_\sigma^\dagger\}= n_{\cc{\sigma}},
$$
$$
[n_{\cc{\sigma}} f_\sigma, H] = \cancelto{\approx 0}{[n_{\cc{\sigma}}, H]} f_\sigma +
n_{\cc{\sigma}} [f_\sigma, H] =
\eps_\sigma n_{\cc{\sigma}} f_\sigma + U n_{\cc{\sigma}}^2 f_\sigma +
\sum_{\alpha \k} V_{\alpha \k}^* n_{\cc{\sigma}} c_{\alpha \k \sigma} =
$$
$$
= \eps_\sigma n_{\cc{\sigma}} f_\sigma + U n_{\cc{\sigma}} f_\sigma +
\sum_{\alpha \k} V_{\alpha \k}^* n_{\cc{\sigma}} c_{\alpha \k \sigma}.
$$
Acima, estamos fazendo a aproximação $[n_{\cc{\sigma}}, H] = [n_{\cc{\sigma}}, H_{\text{C}}] \approx 0$. Usamos também que $n_{\cc{\sigma}}^2 = n_{\cc{\sigma}}$. Temos então
$$
\Big(
\omega^+ - \eps_\sigma - U
\Big) \gf{n_{\cc{\sigma}} f_\sigma}{f_\sigma^\dagger} =
\ev{n_{\cc{\sigma}}} +
\sum_{\alpha \k} V_{\alpha \k}^*
\gf{n_{\cc{\sigma}} c_{\alpha \k \sigma}}{f_\sigma^\dagger}.
$$

Vamos mais fundo de novo, calculando a equação de movimento para $\gf{n_{\cc{\sigma}} c_{\alpha \k \sigma}}{f_\sigma^\dagger}$:
$$
\omega^+ \gf{n_{\cc{\sigma}} c_{\alpha \k \sigma}}{f_\sigma^\dagger} =
\ev{\{n_{\cc{\sigma}} c_{\alpha \k \sigma}, f_\sigma^\dagger\}} +
\gf{[n_{\cc{\sigma}} c_{\alpha \k \sigma}, H]}{f_\sigma^\dagger}.
$$

Os comutadores dão
$$
\{n_{\cc{\sigma}} c_{\alpha \k \sigma}, f_\sigma^\dagger\} =
n_{\cc{\sigma}} \{c_{\alpha \k \sigma}, f_\sigma^\dagger\} = 0,
$$
$$
[n_{\cc{\sigma}} c_{\alpha \k \sigma}, H] =
\cancelto{\approx 0}{[n_{\cc{\sigma}}, H]} c_{\alpha \k \sigma} +
n_{\cc{\sigma}} [c_{\alpha \k \sigma}, H] =
\eps_{\alpha \k} n_{\cc{\sigma}} c_{\alpha \k \sigma} +
V_{\alpha \k} n_{\cc{\sigma}} f_\sigma.
$$

Substituindo na EOM, temos
$$
\gf{n_{\cc{\sigma}} c_{\alpha \k \sigma}}{f_\sigma^\dagger} =
\frac{V_{\alpha \k}}{\omega^+  - \eps_{\alpha \k}}
\gf{n_{\cc{\sigma}} f_\sigma}{f_\sigma^\dagger}.
$$

Agora substituiremos esse resultado na EOM para $\gf{n_{\cc{\sigma}} f_\sigma}{f_\sigma^\dagger}$:
\begin{equation} \label{eq:not-meanfield}
\Big(
\omega^+ - \eps_\sigma - U
\Big) \gf{n_{\cc{\sigma}} f_\sigma}{f_\sigma^\dagger} =
\ev{n_{\cc{\sigma}}} +
\sum_{\alpha \k} \frac{\abs{V_{\alpha \k}}^2}{\omega^+  - \eps_{\alpha \k}}
\gf{n_{\cc{\sigma}} f_\sigma}{f_\sigma^\dagger}
\end{equation}
$$
\implies \gf{n_{\cc{\sigma}} f_\sigma}{f_\sigma^\dagger} =
\frac{\ev{n_{\cc{\sigma}}}}{\omega^+ - \eps_\sigma - U - \Sigma_0(\omega^+)}.
$$

Finalmente, podemos voltar na EOM para $G_\sigma$:
$$
\Big(
\omega^+ - \eps_\sigma - \Sigma_0(\omega^+)
\Big) G_\sigma(\omega^+)
= 1 + \frac{U \ev{n_{\cc{\sigma}}}}{\omega^+ - \eps_\sigma - \Sigma_0(\omega^+) - U}
\implies
$$
$$
G_\sigma(\omega^+) =
\frac{\omega^+ - \eps_\sigma - \Sigma_0(\omega^+) - U (1 - \ev{n_{\cc{\sigma}}})}
{\Big(\omega^+ - \eps_\sigma - \Sigma_0(\omega^+)\Big)\Big(\omega^+ - \eps_\sigma - \Sigma_0(\omega^+) - U\Big)}.
$$

Agora, chame $x = \omega^+ - \eps_\sigma - \Sigma_0(\omega^+)$. Temos que (somando e subtraindo $x \cdot \ev{n_{\cc{\sigma}}}$):
$$
G_\sigma(\omega^+) =
\frac{x - U (1 - \ev{n_{\cc{\sigma}}})}
{x\Big(x - U\Big)} =
\frac{x \ev{n_{\cc{\sigma}}}}{x\Big(x - U\Big)} +
\frac{x - x \ev{n_{\cc{\sigma}} } - U (1 - \ev{n_{\cc{\sigma}}})}
{x\Big(x - U\Big)} =
$$
$$
= \frac{\ev{n_{\cc{\sigma}}}}{{x - U}} + \frac{1 - \ev{n_{\cc{\sigma}}}}{x}.
$$

Assim, chegamos finalmente à equação (11) do paper.
$$
\boxed{
G_\sigma(\omega^+) =
\frac{1 - \ev{n_{\cc{\sigma}}}}{\omega^+ - \eps_\sigma - \Sigma_0(\omega^+)} +
\frac{\ev{n_{\cc{\sigma}}}}{{\omega^+ - \eps_\sigma - U - \Sigma_0(\omega^+)}}.
}
$$

\section{Aproximação EOM1}

Agora justificarei a aproximação EOM1 (equação (15) do paper). Ela se baseia na aproximação de Hubbard-I. Lembre que a equação exata para $\gf{n_{\cc{\sigma}} f_\sigma}{f_\sigma^\dagger}$ era
$$
\omega^+ \gf{n_{\cc{\sigma}} f_\sigma}{f_\sigma^\dagger} =
\ev{\{n_{\cc{\sigma}} f_\sigma, f_\sigma^\dagger\}} +
\gf{[n_{\cc{\sigma}} f_\sigma, H]}{f_\sigma^\dagger},
$$
onde $\{n_{\cc{\sigma}} f_\sigma, f_\sigma^\dagger\} = n_{\cc{\sigma}}$. Para o comutador, tínhamos que $[n_{\cc{\sigma}} f_\sigma, H] = [n_{\cc{\sigma}}, H] f_\sigma + n_{\cc{\sigma}} [f_\sigma, H]$, onde fizemos a aproximação $[n_{\cc{\sigma}}, H] \approx 0$. Desta vez, fiquemos com esse termo mais um pouco:
$$
\Big(
\omega^+ - \eps_\sigma - U
\Big) \gf{n_{\cc{\sigma}} f_\sigma}{f_\sigma^\dagger} =
\ev{n_{\cc{\sigma}}} +
\sum_{\alpha \k} V_{\alpha \k}^*
\gf{n_{\cc{\sigma}} c_{\alpha \k \sigma}}{f_\sigma^\dagger} +
\textcolor{MidnightBlue}{\gf{[n_{\cc{\sigma}}, H] f_\sigma}{f_\sigma^\dagger}}.
$$

Calculando $[n_{\cc{\sigma}}, H]$ explicitamente, temos
$$
[n_{\cc{\sigma}}, H] = [n_{\cc{\sigma}}, H_{\text{C}}] =
\sum_{\alpha \k \tau} \Big\{
V_{\alpha \k} [n_{\cc{\sigma}}, c_{\alpha \k \tau}^\dagger f_\tau] +
V_{\alpha \k}^* [n_{\cc{\sigma}}, f_\tau^\dagger c_{\alpha \k \tau}]
\Big\}.
$$
Temos que
$$
[n_{\cc{\sigma}}, c_{\alpha \k \tau}^\dagger f_\tau] =
\cancelto{0}{[n_{\cc{\sigma}}, c_{\alpha \k \tau}^\dagger]} f_\tau +
c_{\alpha \k \tau}^\dagger
[n_{\cc{\sigma}}, f_\tau] =
- \delta_{\tau \cc{\sigma}} \, c_{\alpha \k \tau}^\dagger f_{\tau},
$$
$$
[n_{\cc{\sigma}}, f_\tau^\dagger c_{\alpha \k \tau}] =
[n_{\cc{\sigma}}, f_\tau^\dagger] c_{\alpha \k \tau} +
f_\tau^\dagger \cancelto{0}{[n_{\cc{\sigma}}, c_{\alpha \k \tau}]} =
+ \delta_{\tau \cc{\sigma}} \, f_{\tau}^\dagger c_{\alpha \k \tau}.
$$

Isso nos dá
$$
[n_{\cc{\sigma}}, H] =
\sum_{\alpha \k} \Big\{
- V_{\alpha \k} c_{\alpha \k \cc{\sigma}}^\dagger f_{\cc{\sigma}} +
V_{\alpha \k}^* f_{\cc{\sigma}}^\dagger c_{\alpha \k \cc{\sigma}}
\Big\}.
$$

Substituindo na EOM:
$$
\Big(
\omega^+ - \eps_\sigma - U
\Big) \gf{n_{\cc{\sigma}} f_\sigma}{f_\sigma^\dagger} =
\ev{n_{\cc{\sigma}}} +
\sum_{\alpha \k} V_{\alpha \k}^*
\gf{n_{\cc{\sigma}} c_{\alpha \k \sigma}}{f_\sigma^\dagger} +
$$
$$
\textcolor{MidnightBlue}{
\sum_{\alpha \k} \Big\{
- V_{\alpha \k} \gf{c_{\alpha \k \cc{\sigma}}^\dagger f_{\cc{\sigma}} f_\sigma}
{f_\sigma^\dagger}
+ V_{\alpha \k}^* \gf{f_{\cc{\sigma}}^\dagger c_{\alpha \k \cc{\sigma}} f_\sigma}
{f_\sigma^\dagger}
\Big\}}.
$$

A aproximação Hubbard-I consiste de \textbf{três} aproximações. As \textbf{duas} primeiras (equações (46) e (47) de \texttt{hubbard1963}) são
$$
\gf{c_{\alpha \k \cc{\sigma}}^\dagger f_{\cc{\sigma}} f_\sigma}{f_\sigma^\dagger}
\approx
\ev{c_{\alpha \k \cc{\sigma}}^\dagger f_{\cc{\sigma}}} \gf{f_\sigma}{f_\sigma^\dagger}
=
\ev{c_{\alpha \k \cc{\sigma}}^\dagger f_{\cc{\sigma}}} G_\sigma(\omega^+).
$$
$$
\gf{f_{\cc{\sigma}}^\dagger c_{\alpha \k \cc{\sigma}} f_\sigma}{f_\sigma^\dagger}
\approx
\ev{f_{\cc{\sigma}}^\dagger c_{\alpha \k \cc{\sigma}}} \gf{f_\sigma}{f_\sigma^\dagger}
=
\ev{f_{\cc{\sigma}}^\dagger c_{\alpha \k \cc{\sigma}}} G_\sigma(\omega^+).
$$

Com isso, o termo colorido fica
$$
\textcolor{MidnightBlue}{
\sum_{\alpha \k} \Big\{
- V_{\alpha \k} \gf{c_{\alpha \k \cc{\sigma}}^\dagger f_{\cc{\sigma}} f_\sigma}
{f_\sigma^\dagger}
+ V_{\alpha \k}^* \gf{f_{\cc{\sigma}}^\dagger c_{\alpha \k \cc{\sigma}} f_\sigma}
{f_\sigma^\dagger}
\Big\}} =
$$
$$
\textcolor{MidnightBlue}{
\sum_{\alpha \k} \Big\{
- V_{\alpha \k} \ev{c_{\alpha \k \cc{\sigma}}^\dagger f_{\cc{\sigma}}}
+ V_{\alpha \k}^* \ev{f_{\cc{\sigma}}^\dagger c_{\alpha \k \cc{\sigma}}}
\Big\}  \, G_\sigma(\omega^+) } = \textcolor{MidnightBlue}{T} \, G_\sigma(\omega^+).
$$

No \texttt{hubbard1963} ele argumenta que o termo $T$ é nulo por simetria translacional. No nosso caso, o SIAM é somente de uma impureza. Mas se assumirmos que existissem várias impurezas regularmente dispostas, podemos utilizar o argumento de simetria translacional também (eu acho). Nesse caso, temos que $T = 0$.

Mas observe o seguinte: colocar $\textcolor{MidnightBlue}{T} = 0$ implica que o termo que vem de $\textcolor{MidnightBlue}{[n_{\cc{\sigma}}, H]}$ é \textbf{nulo}. Portanto, ainda não estamos fazendo nada de novo em relação ao EOM0. Nós ainda temos a mesma equação de movimento
$$
\Big(
\omega^+ - \eps_\sigma - U
\Big) \gf{n_{\cc{\sigma}} f_\sigma}{f_\sigma^\dagger} =
\ev{n_{\cc{\sigma}}} +
\sum_{\alpha \k} V_{\alpha \k}^*
\gf{n_{\cc{\sigma}} c_{\alpha \k \sigma}}{f_\sigma^\dagger}.
$$

A \textbf{última} aproximação do Hubbard-I (equação (45) de \texttt{hubbard1963}) é
$$
\gf{n_{\cc{\sigma}} c_{\alpha \k \cc{\sigma}}}{f_\sigma^\dagger}
\approx
\ev{n_{\cc{\sigma}}} \gf{c_{\alpha \k \cc{\sigma}}}{f_\sigma^\dagger}.
$$

Calculando rapidamente a equação de movimento para $\gf{c_{\alpha \k \cc{\sigma}}}{f_\sigma^\dagger}$, obtemos facilmente que
$$
\gf{c_{\alpha \k \cc{\sigma}}}{f_\sigma^\dagger} =
\frac{V_{\alpha \k}}{\omega^+ - \eps_{\alpha \k}} \, G_\sigma(\omega^+).
$$

Com isso, a equação para $\gf{n_{\cc{\sigma}} f_\sigma}{f_\sigma^\dagger}$ fica:
\begin{equation} \label{eq:yes-meanfield}
\Big(
\omega^+ - \eps_\sigma - U
\Big) \gf{n_{\cc{\sigma}} f_\sigma}{f_\sigma^\dagger} =
\ev{n_{\cc{\sigma}}} +
\sum_{\alpha \k} \frac{\abs{V_{\alpha \k}}^2}
{\omega^+ - \eps_{\alpha \k}} \ev{n_{\cc{\sigma}}} G_\sigma(\omega^+).
\end{equation}

Compare a equação \ref{eq:yes-meanfield} acima com a equação \ref{eq:not-meanfield}. É como se tivéssemos aplicado na equação \ref{eq:not-meanfield} uma aproximação de \textit{mean-field} $\gf{n_{\cc{\sigma}} f_\sigma}{f_\sigma^\dagger} \approx \ev{n_{\cc{\sigma}}} G_\sigma(\omega^+)$, obtendo assim a equação \ref{eq:yes-meanfield}. Isso mostra que na realidade, $\text{EOM0} > \text{EOM1}$ é um resultado mais exato, não o contrário.

Por fim, temos em EOM1:
$$
\gf{n_{\cc{\sigma}} f_\sigma}{f_\sigma^\dagger} =
\ev{n_{\cc{\sigma}}} \,
\frac{1 + \Sigma_0(\omega^+) G_\sigma(\omega^+)}{\omega^+ - \eps_\sigma - U},
$$
o que nos dá a equação (15) do paper para o EOM1
$$
\boxed{
G_\sigma(\omega^+) =
\frac{\omega^+ - \eps_\sigma - U(1-\ev{n_{\cc{\sigma}}})}
{(\omega^+ - \eps_\sigma)(\omega^+ - \eps_\sigma - U) -
\Sigma_0(\omega^+)
\big[
\omega^+ - \eps_\sigma - U(1-\ev{n_{\cc{\sigma}}})
\big]
}. }
$$

\pagebreak

\section{Sympy}

$$
\boxed{
G_\sigma(\omega^+) =
\frac{1 - \ev{n_{\cc{\sigma}}}}{\omega^+ - \eps_\sigma - \Sigma_0(\omega^+)} +
\frac{\ev{n_{\cc{\sigma}}}}{{\omega^+ - \eps_\sigma - U - \Sigma_0(\omega^+)}}.
}
$$

Escrevendo $\Sigma_0(\omega) = \Lambda(\omega) - i \Delta(\omega)$, temos
$$
\boxed{
A_\sigma(\omega) = \frac{- \Im{G_\sigma(\omega^+)}}{\pi} =
\frac{\Delta(\omega) \langle n_{\overline{\sigma}} \rangle / \pi}
{\left[\omega - \epsilon_{\sigma} - U - \Lambda(\omega)\right]^{2}+\Delta(\omega)^{2}}+\frac{\Delta(\omega) \left(1 - \langle n_{\overline{\sigma}} \rangle\right) / \pi}
{\left[\omega - \epsilon_{\sigma} - \Lambda(\omega)\right]^{2} + \Delta(\omega)^{2}}.
}
$$


\end{document}
