\documentclass[a4paper,fleqn,12pt]{article}
%\usepackage[T1]{fontenc}
\usepackage[brazilian]{babel}
%\usepackage[utf8]{inputenc}
%\usepackage[left=2.5cm,right=2.5cm,top=3cm,bottom=2.5cm]{geometry}
%\usepackage{mathtools}
%\usepackage{amsthm}
%\usepackage{amsmath}
%\usepackage{nccmath}
%\usepackage{amssymb}
%\usepackage{amsfonts}
\usepackage{physics}
%\usepackage{dsfont}
%\usepackage{mathrsfs}

\usepackage{titling}
\usepackage{indentfirst}

\usepackage{bm}
\usepackage[dvipsnames]{xcolor}
\usepackage{cancel}

\usepackage{xurl}
\usepackage[colorlinks=true]{hyperref}

% code
\definecolor{bg}{rgb}{0.90,0.90,0.90}
\usepackage{minted}

%\usepackage{float}
%\usepackage{graphicx}
%\usepackage{tikz}
%\usepackage{caption}
%\usepackage{subcaption}

%%%%%%%%%%%%%%%%%%%%%%%%%%%%%%%%%%%%%%%%%%%%%%%%%%%

\newcommand{\eps}{\epsilon}
\renewcommand{\a}{\alpha}
\newcommand{\vphi}{\varphi}
\newcommand{\cte}{\text{cte}}

\newcommand{\N}{\mathbb{N}}
\newcommand{\Z}{\mathbb{Z}}
\newcommand{\Q}{\mathbb{Q}}
\newcommand{\R}{\mathbb{R}}
\newcommand{\C}{\mathbb{C}}
\renewcommand{\H}{\mathcal{H}}
\newcommand{\intR}{\int_{-\infty}^{\infty}}

\newcommand{\E}{\vb{E}}
\newcommand{\B}{\vb{B}}
\renewcommand{\v}{\vb{v}}
\renewcommand{\r}{\vb{r}}
\renewcommand{\k}{\vb{k}}
\newcommand{\p}{\vb{p}}
\newcommand{\q}{\vb{q}}
\newcommand{\F}{\vb{F}}

\newcommand{\gf}[2]{\ev{\ev{#1 : #2}}}
\newcommand{\zub}[2]{\ev{\comm{#1}{#2}_\mp}}
\newcommand{\up}{\uparrow}
\newcommand{\dn}{\downarrow}
\newcommand{\0}{\ket{0}}
\newcommand{\2}{\ket{2}}
\newcommand{\kup}{\ket{\uparrow}}
\newcommand{\kdn}{\ket{\downarrow}}
%\newcommand{\bup}{\bra{\uparrow}}
%\newcommand{\bdn}{\bra{\downarrow}}

\renewcommand{\tr}{\text{tr}}
\renewcommand{\a}{\alpha}
\newcommand{\s}{\sigma}
\newcommand{\w}{\omega}
\newcommand{\us}{{\overline{\sigma}}}
\renewcommand{\d}{\dagger}
\newcommand{\nF}{n_\text{F}}

\renewcommand{\S}[1]{\mathcal{#1}}
%\newcommand{\prodint}[2]{\left\langle #1 , #2 \right\rangle}
\newcommand{\cc}[1]{\overline{#1}}
\newcommand{\Eval}[3]{\eval{\left( #1 \right)}_{#2}^{#3}}

\newcommand{\unit}[1]{\; \mathrm{#1}}

\newcommand{\n}{\medskip}
\newcommand{\e}{\quad \mathrm{e} \quad}
\newcommand{\ou}{\quad \mathrm{ou} \quad}
\newcommand{\virg}{\, , \;}
\newcommand{\ptodo}{\forall \,}
\renewcommand{\implies}{\; \Rightarrow \;}
%\newcommand{\eqname}[1]{\tag*{#1}} % Tag equation with name

% math %
\renewcommand{\erf}[1]{\text{erf}\left(#1\right)}
\newcommand{\floor}[1]{\left\lfloor #1 \right\rfloor}
\newcommand{\ceil}[1]{\left\lceil #1 \right\rceil}

\setlength{\droptitle}{-6em}


\title{\Huge{\textbf{Gagaumesh}}}
\author{Mateus Marques}

\begin{document}

\maketitle

\section{Erroing the gaussian}

A gaussian is defined as
$$
g(x, \mu, \sigma) = \frac{1}{\sigma \sqrt{2\pi}} \exp{-\frac{(x-\mu)^2}{2 \sigma^2}}.
$$

The error function erf is given by ($x = \mu + t\sigma \sqrt{2}$).
$$
\erf{z} = \frac{2}{\sqrt{\pi}} \int_0^z e^{-t^2} \dd{t} =
\frac{2}{\sqrt{\pi}} \int_\mu^{\mu + z \sigma \sqrt{2}}
e^{-\frac{(x-\mu)^2}{2\sigma^2}} \frac{\dd{x}}{\sigma \sqrt{2}} \implies
$$
$$
\boxed{
\erf{z} = 2 \int_\mu^{\mu+z\sigma\sqrt{2}} g(x, \mu, \sigma) \dd{x}. }
$$
$$
\boxed{
\erf{\frac{w-\mu}{\sigma\sqrt{2}}} = 2 \int_\mu^{w} g(x, \mu, \sigma) \dd{x}. }
$$
$$
\boxed{
\dv{z} \erf{z} = \sigma \, 2\sqrt{2} \, g(\mu + z\sigma\sqrt{2}, \mu, \sigma).}
$$


\section{\texttt{class Gargaumesh}}

\begin{minted}[bgcolor=bg]{py}
class GauMesh:
    def __init__(self, A, alpha, xpos, xmin, xmax):
        self.ng = len(A)
        self.A = A
        self.alpha = alpha
        self.xpos = xpos
        self.xmin = xmin
        self.xmax = xmax
        self.Npoints = floor(self.N(self.xmax) - self.N(self.xmin))
    def N(self, x):
        val = 0.0
        for i in range(self.ng):
            Ai = self.A[i]
            ai = self.alpha[i]
            xi = self.xpos[i]
            val += 0.5*Ai*sqrt(pi/ai)*erf(sqrt(ai)*(x-xi))
        return val
\end{minted}

\pagebreak

\section{Unraveling the spaghetti}

\mintinline{py}{int ng} $\equiv N_g$ = ``number of Gaussians''.

\mintinline{py}{float fwmh[ng]} $\equiv F_i$ =  ``full width at half maximum''.

\mintinline{py}{float alpha[ng]} $\equiv \alpha_i$ = ``Gaussian exponents''.

\mintinline{py}{float A[ng]} $\equiv A_i$ = ``weights of Gaussians''.

\mintinline{py}{float xpos[ng]} $\equiv \mu_i$ = ``position/center/mean of the Gaussians''.

\mintinline{py}{float xmin} $\equiv x_{\text{min}}$ = ``start of mesh''.

\mintinline{py}{float xmax} $\equiv x_{\text{max}}$ = ``end of mesh''.

\mintinline{py}{float dx0[ng]} $\equiv dx_i$ = ``maximal resolution of each Gaussian mesh near its center''.

\n

What ``full width at half maximum'' means is $F_i = 2 w_i$, where
$$
e^{- \frac{w_i^2}{2\sigma_i^2}} = 1/2 \implies
\boxed{ F_i = \sigma_i \sqrt{8 \log 2}. }
$$

``Gaussian exponent'' $\alpha_i$ is such that
$$
\sigma_i = \frac{1}{\sqrt{2\alpha_i}} \implies
\boxed{ g(x, \mu_i, \sigma_i) \propto
e^{- \frac{(x-\mu_i)^2}{2\sigma_i^2}} = e^{-\alpha_i (x - \mu_i)^2}. }
$$

Therefore, we have
$$
\erf{\sqrt{\alpha_i}\,(x-\mu_i)} = \erf{\frac{x-\mu_i}{\sigma_i\sqrt{2}}} =
2 \int_{\mu_i}^x g(t, \mu_i, \sigma_i) \dd{t}.
$$
$$
\erf{\sqrt{\alpha_i} \, dx_i/2} = \erf{\frac{dx_i/2}{\sigma_i\sqrt{2}}} =
2 \int_{\mu_i}^{\mu_i + dx_i/2} g(t, \mu_i, \sigma_i) \dd{t}.
$$

\n

We have the following code inside \mintinline{py}{main()}:
\begin{minted}[bgcolor=bg]{py}
for i in range(ng):
    alpha.append(4*log(2)/(fwhm[i])**2)
    A.append(2*sqrt(alpha[i]/pi)/erf(sqrt(alpha[i])*dx0[i]/2))
\end{minted}

$$
\alpha_i = \frac{4 \log 2}{F_i^2} \implies
\boxed{ \sigma_i = \frac{1}{\sqrt{2\alpha_i}} = \frac{F_i}{\sqrt{8 \log 2}}. }
$$
$$
A_i = \frac{2 \sqrt{\alpha_i / \pi}}{\erf{\sqrt{\alpha_i} \, dx_i / 2}} =
2 \cdot \frac{1}{\sigma_i \sqrt{2\pi}} \cdot
\frac{1}{\erf{\sqrt{\alpha_i} \, dx_i / 2}}.
$$

The \mintinline{py}{N(self, x)} method inside \mintinline{py}{GauMesh}:
\begin{minted}[bgcolor=bg]{py}
def N( self, x ):
    val = 0.0
    for i in range(self.ng):
        Ai = self.A[i]
        ai = self.alpha[i]
        xi = self.xpos[i]
        val += 0.5*Ai*sqrt(pi/ai)*erf(sqrt(ai)*(x-xi))
    return val
\end{minted}

The function \mintinline{py}{N(self, x)} is given by $\aleph(x)$, where
$$
\aleph(x) = \sum_{i = 1}^{N_g} \frac{A_i}{2} \, \sqrt{\frac{\pi}{\alpha_i}} \,
\erf{\sqrt{\alpha_i} \, (x - \mu_i)} =
\sum_{i=1}^{N_g}
\frac{\erf{\sqrt{\alpha_i} \, (x - \mu_i)}}{\erf{\sqrt{\alpha_i} \, dx_i / 2}} =
\sum_{i = 1}^{N_g} f_i(x).
$$

Let $f_i(x)$ be
$$
f_i(x) = \frac{\erf{\sqrt{\alpha_i} \, (x - \mu_i)}}
{\erf{\sqrt{\alpha_i} \, dx_i / 2}}.
$$

If we take $x_i^\pm = \mu_i \pm dx_i/2$, we get $f(x_i^\pm) = \pm 1$. This is what is meant by $dx_i$ determining the maximum resolution.

We then have

$$
\aleph(x) =
\sum_{i=1}^{N_g}
\frac{\int_{\mu_i}^x g(t, \mu_i, \sigma_i) \dd{t}}
{\int_{\mu_i}^{\mu_i + dx_i/2} g(y, \mu_i, \sigma_i) \dd{y}}
\quad \text{and} \quad
\dv{\aleph}{x} =
\sum_{i=1}^{N_g}
\frac{g(x, \mu_i, \sigma_i)}
{\int_{\mu_i}^{\mu_i + dx_i/2} g(y, \mu_i, \sigma_i) \dd{y}}.
$$

The number of points of the mesh is given by
$$
N_{\text{points}} = \floor{\aleph(x_{\text{max}}) - \aleph(x_{\text{min}})},
$$
which is $N_{\text{points}} = \floor{2 \, \aleph(x_{\text{max}})}$ if $x_{\text{min}} = - x_{\text{max}}$.

Also
$$
N_{\text{points}} \approx
\int_{-\infty}^{\infty} \dv{\aleph}{x} \dd{x} =
\sum_{i=1}^{N_g} \frac{1}
{\int_{\mu_i}^{\mu_i + dx_i/2} g(y, \mu_i, \sigma_i) \dd{y}}.
$$
But since $\sqrt{\frac{\pi}{4\log 2}} = 1.0647 \approx 1$
$$
\frac{1}{\int_{\mu_i}^{\mu_i + dx_i/2} g(y, \mu_i, \sigma_i) \dd{y}}
\approx \frac{2 \sigma_i \sqrt{2\pi}}{dx_i} =
\frac{2 F_i}{dx_i} \, \sqrt{\frac{\pi}{4\log 2}} \approx \frac{2 F_i}{dx_i},
$$
it follows
$$
\boxed{
N_{\text{points}} \approx \sum_{i=1}^{N_g} \frac{2 F_i}{dx_i}. }
$$

\section{Fixing the loop}

Previously we had
\begin{minted}[bgcolor=bg]{py}
# original script in Python2 and just wrong
xn0 = xmin
format = '%15.10f'
for i in range(Ntot):
    ni=i-Ntot/2
    xn=bisect( meshN, xn0, xmax, (), 1e-15, 10000 )
    if ni != 0:
        print format % xn
    xn0=xn
\end{minted}

Now we have
\begin{minted}[bgcolor=bg]{py}
# after being fixed and translated to Python3
N_points = mesh.Npoints
Func_min = mesh.N(xmin)
# we define Func = mesh.N(x) - mesh.N(xmin)
# in order for Func(xmin) = 0 and ceil( Func(xmax) ) = N_points
Func = lambda x, n: mesh.N(x) - Func_min - n
print("%15.10f" % xmin) # We always print xmin
xn0 = xmin
for n in range(1, N_points + 1):
    xn = find_root(Func, xn0, xmax, args=(n), xtol=1e-15, maxiter=10000)
    # find_root is actually scipy.optimize.brentq
    # the best bracket root finder
    print("%15.10f" % xn)
    xn0 = xn
print("%15.10f" % xmax) # We always print xmax
\end{minted}

The algorithm is as follows:
\begin{itemize}
\item First, define $F(x) = \aleph(x) - \aleph(x_{\text{min}})$.
\item We want the interval $x \in [x_{\text{min}}, x_{\text{max}}]$.
Thus, $F(x) \in [0, \aleph(x_{\text{max}}) - \aleph(x_{\text{min}})]$, where
$N_{\text{points}} = \floor{\aleph(x_{\text{max}}) - \aleph(x_{\text{min}}}$.
\item We then calculate the roots $x_n$ of $F$ at all the integers from $n = 1, 2, \ldots, N_{\text{points}}$ and \mintinline{py}{print(x_n)}. This is equivalent of calculating the inverse of $F$ at the integers, and will give $x$'s gaussian spaced (hence gaussian mesh).
\item We then update \mintinline{py}{xn0 = xn}, because \mintinline{py}{xn0} was the root of $F_{n-1}(x) = F(x) - (n-1)$ and the next is \mintinline{py}{xn > xn0}, root of $F_n(x) = F(x) - n$.
\end{itemize}

\end{document}
