\documentclass[a4paper,fleqn,12pt]{article}
\usepackage[T1]{fontenc}
%\usepackage[utf8]{inputenc}
\usepackage[brazilian]{babel}
\usepackage[left=2.5cm,right=2.5cm,top=3cm,bottom=2.5cm]{geometry}
\usepackage{mathtools}
%\usepackage{amsthm}
\usepackage{amsmath}
\usepackage{nccmath}
\usepackage{amssymb}
\usepackage{amsfonts}
\usepackage{physics}
\usepackage{dsfont}
%\usepackage{mathrsfs}

\usepackage{titling}
\usepackage{indentfirst}

\usepackage{bm}
\usepackage[dvipsnames]{xcolor}
\usepackage{cancel}

\usepackage{xurl}
\usepackage[colorlinks=true]{hyperref}

\usepackage{float}
\usepackage{graphicx}
\usepackage{tikz}
\usepackage{caption}
\usepackage{subcaption}

%%%%%%%%%%%%%%%%%%%%%%%%%%%%%%%%%%%%%%%%%%%%%%%%%%%

\newcommand{\eps}{\epsilon}
\newcommand{\vphi}{\varphi}
\newcommand{\cte}{\text{cte}}

\newcommand{\N}{\mathbb{N}}
\newcommand{\Z}{\mathbb{Z}}
\newcommand{\Q}{\mathbb{Q}}
\newcommand{\R}{\mathbb{R}}
\newcommand{\C}{\mathbb{C}}
\renewcommand{\H}{\hat{H}}
\newcommand{\intR}{\int_{-\infty}^{\infty}}

\newcommand{\0}{\vb{0}}
\newcommand{\1}{\mathds{1}}
\newcommand{\E}{\vb{E}}
\newcommand{\B}{\vb{B}}
\renewcommand{\v}{\vb{v}}
\renewcommand{\r}{\vb{r}}
\renewcommand{\k}{\vb{k}}
\newcommand{\p}{\vb{p}}
\newcommand{\q}{\vb{q}}
\newcommand{\F}{\vb{F}}

\renewcommand{\a}{\hat{a}}
\renewcommand{\b}{\hat{b}}
\renewcommand{\c}{\hat{c}}
\newcommand{\nn}{\hat{n}}

\newcommand{\gf}[2]{\ev{\ev{#1 : #2}}}
\newcommand{\zub}[2]{\ev{\comm{#1}{#2}_\mp}}

\newcommand{\s}[1]{\mathcal{#1}}
%\newcommand{\prodint}[2]{\left\langle #1 , #2 \right\rangle}
\newcommand{\cc}[1]{\overline{#1}}
\newcommand{\Eval}[3]{\eval{\left( #1 \right)}_{#2}^{#3}}

\newcommand{\unit}[1]{\; \mathrm{#1}}

\newcommand{\n}{\medskip}
\newcommand{\e}{\quad \mathrm{e} \quad}
\newcommand{\ou}{\quad \mathrm{ou} \quad}
\newcommand{\virg}{\, , \;}
\newcommand{\ptodo}{\forall \,}
\renewcommand{\implies}{\; \Rightarrow \;}
%\newcommand{\eqname}[1]{\tag*{#1}} % Tag equation with name

\setlength{\droptitle}{-8em}


\title{\Huge{\textbf{Anderson}}}
\author{Mateus Marques}

\begin{document}

\maketitle

\section{Modelo}

Nós tínhamos exatamente
$$
\qty[
\omega^+ - \eps_0 - \Sigma^{(0)}(\omega^+)
] G_{d\sigma}(\omega^+) = 1 + U \cdot
D_{d\sigma}(\omega^+),
$$
onde, pela aproximação de \textit{mean-field}
$$
\Sigma^{(0)}(\omega^+) = \sum_{\k} \frac{\abs{t_{\k}}^2}{\omega^+ - \eps_{\k}}
\e
D_{d\sigma}(\omega^+) = \gf{\nn_{d\cc{\sigma}} \, \c_{d\sigma}}{\c^\dagger_{d\sigma}}
\approx \ev{\nn_{d\cc{\sigma}}} G_{d\sigma}(\omega^+).
$$

Portanto
$$
G_{d\sigma}(\omega^+) =
\frac{1}{\omega^+ - \eps_0 - \Sigma^{(0)}(\omega^+) - U \ev{\nn_{d\cc{\sigma}}}} =
$$
$$
\frac{1}{
\qty(\omega - \eps_0 - U \ev{\nn_{d\cc{\sigma}}} - \Re{\Sigma^{(0)}(\omega^+)})
+ i \, \qty(\eta - \Im{\Sigma^{(0)}(\omega^+)})
}.
$$
e
$$
\boxed{ \Im{G_{d\sigma}(\omega^+)} =
\frac{- \qty[\eta - \Im{\Sigma^{(0)}(\omega^+)}]}{
\qty(\omega - \eps_0 - U \ev{\nn_{d\cc{\sigma}}} - \Re{\Sigma^{(0)}(\omega^+)})^2
+ \qty(\eta - \Im{\Sigma^{(0)}(\omega^+)})^2}. }
$$

\n

Podemos então calcular $\Sigma^{(0)}(\omega^+)$ por
$$
\Sigma^{(0)}(\omega^+) = \sum_{\k} \frac{\abs{t_{\k}}^2}{\omega^+ - \eps_{\k}} =
\text{Vol} \cdot \int \frac{\dd[d]{{\k}}}{(2\pi)^d} \, \frac{\abs{t({\k})}^2}{\omega^+ - \eps({\k})}.
$$
Supondo que $t(k)$ e $\eps(k)$ só dependem do módulo $k = \abs{\k}$, temos
$$
\Sigma^{(0)}(\omega^+) = \frac{\text{Vol}}{(2\pi)^d}
\int \dd{\Omega}
\int \frac{\abs{t(k)}^2 \dd{k}}{\omega^+ - \eps(k)}.
$$
Sendo então que $\dd{k} = d(\eps) \dd{\eps}$ e $\omega^+ = \omega + i\eta$, temos
$$
\Sigma^{(0)}(\omega^+) = \frac{\Omega \cdot \text{Vol}}{(2\pi)^d}
\int \frac{\abs{t(\eps)}^2}{\omega - \eps + i \eta} \, d(\eps) \dd{\eps}
$$
$$
= \frac{\Omega \cdot \text{Vol}}{(2\pi)^d} \, \qty{
P.V. \int \frac{\abs{t(\eps)}^2}{\omega - \eps} \, d(\eps) \dd{\eps}
- i \pi \int \delta(\omega - \eps) \, \abs{t(\eps)}^2  d(\eps) \dd{\eps}
}
$$
Definindo então $\Delta(\eps) = \pi \, \frac{\Omega \cdot \text{Vol}}{(2\pi)^d} \, \abs{t(\eps)}^2  d(\eps)$, temos que
$$
\boxed{ \Sigma^{(0)}(\omega^+) = P.V. \int \frac{\Delta(\eps)}{\omega - \eps} \dd{\eps} - i \Delta(\omega). }
$$

Chamando $\boxed{ \Lambda(\omega) = \Re{\Sigma^{(0)}(\omega^+)} = P.V. \int \frac{\Delta(\eps)}{\omega - \eps} \dd{\eps} }$ e tomando $\eta \to 0^+$, temos
$$
A_{d\sigma}(\omega) = - \frac{1}{\pi} \Im{G_{d\sigma}(\omega^+)} \implies
$$
$$
\boxed{ A_{d\sigma}(\omega) = \frac{\Delta(\omega) / \pi}{
\Big[\omega - \eps_0 - U \ev{\nn_{d\cc{\sigma}}} - \Lambda(\omega) \Big]^2
+ \Big[\Delta(\omega)\Big]^2 }. }
$$

Para temperatura $T = 1/(k_B \beta)$, temos $n_F(\omega) = (e^{\beta \omega} + 1)^{-1}$ e então
$$
\boxed{ \ev{\nn_{d\sigma}} = \intR \frac{A_{d\sigma}(\omega)}{e^{\beta\omega} + 1} \dd{\omega}. }
$$

No caso especial onde $T = 0$ ($\beta = \infty$), temos que $e^{\beta \omega} = +\infty \cdot \theta(\omega)$, o que nos dá
$$
\boxed{ \ev{\nn_{d\sigma}} = \int_{-\infty}^0 A_{d\sigma}(\omega) \dd{\omega}, \quad T = 0. }
$$

Como $A_{d\sigma}$ depende de $\ev{\nn_{d\cc{\sigma}}}$, as equações acima são de \textbf{ponto fixo}
$$
\boxed{ \ev{\nn_{d\sigma}} = \s{F}\qty{\ev{\nn_{d\cc{\sigma}}}} } \text{ , onde } \,
\s{F}\qty{\ev{\nn_{d\sigma}}} =
\intR \frac{A_{d\sigma}(\omega, \ev{\nn_{d\cc{\sigma}}})}{e^{\beta\omega} + 1} \dd{\omega}.
$$

Por enquanto escolheremos
$$
\Delta(\omega) = \Delta_0 \, \qty[1 - \qty(\frac{\omega}{D})^2].
$$


\end{document}
