\documentclass[a4paper,fleqn,12pt]{article}
\usepackage[T1]{fontenc}
%\usepackage[utf8]{inputenc}
\usepackage[brazilian]{babel}
\usepackage[left=2.5cm,right=2.5cm,top=3cm,bottom=2.5cm]{geometry}
\usepackage{mathtools}
%\usepackage{amsthm}
\usepackage{amsmath}
\usepackage{nccmath}
\usepackage{amssymb}
\usepackage{amsfonts}
\usepackage{physics}
\usepackage{dsfont}
%\usepackage{mathrsfs}

\usepackage{titling}
\usepackage{indentfirst}

\usepackage{bm}
\usepackage[dvipsnames]{xcolor}
\usepackage{cancel}

\usepackage{xurl}
\usepackage[colorlinks=true]{hyperref}

\usepackage{float}
\usepackage{graphicx}
\usepackage{tikz}
\usepackage{caption}
\usepackage{subcaption}

%%%%%%%%%%%%%%%%%%%%%%%%%%%%%%%%%%%%%%%%%%%%%%%%%%%

\newcommand{\eps}{\epsilon}
\newcommand{\vphi}{\varphi}
\newcommand{\cte}{\text{cte}}

\newcommand{\N}{\mathbb{N}}
\newcommand{\Z}{\mathbb{Z}}
\newcommand{\Q}{\mathbb{Q}}
\newcommand{\R}{\mathbb{R}}
\newcommand{\C}{\mathbb{C}}
\renewcommand{\H}{\hat{H}}
\newcommand{\intR}{\int_{-\infty}^{\infty}}

\newcommand{\0}{\vb{0}}
\newcommand{\1}{\mathds{1}}
\newcommand{\E}{\vb{E}}
\newcommand{\B}{\vb{B}}
\renewcommand{\v}{\vb{v}}
\renewcommand{\r}{\vb{r}}
\renewcommand{\k}{\vb{k}}
\newcommand{\p}{\vb{p}}
\newcommand{\q}{\vb{q}}
\newcommand{\F}{\vb{F}}

\renewcommand{\a}{\hat{a}}
\renewcommand{\b}{\hat{b}}
\renewcommand{\c}{\hat{c}}
\newcommand{\nn}{\hat{n}}

\newcommand{\gf}[2]{\ev{\ev{#1 : #2}}}
\newcommand{\zub}[2]{\ev{\comm{#1}{#2}_\mp}}

\newcommand{\s}[1]{\mathcal{#1}}
%\newcommand{\prodint}[2]{\left\langle #1 , #2 \right\rangle}
\newcommand{\cc}[1]{\overline{#1}}
\newcommand{\Eval}[3]{\eval{\left( #1 \right)}_{#2}^{#3}}

\newcommand{\unit}[1]{\; \mathrm{#1}}

\newcommand{\n}{\medskip}
\newcommand{\e}{\quad \mathrm{e} \quad}
\newcommand{\ou}{\quad \mathrm{ou} \quad}
\newcommand{\virg}{\, , \;}
\newcommand{\ptodo}{\forall \,}
\renewcommand{\implies}{\; \Rightarrow \;}
%\newcommand{\eqname}[1]{\tag*{#1}} % Tag equation with name

\setlength{\droptitle}{-8em}


\title{\Huge{\textbf{NCA}}}
\author{Mateus Marques}

\renewcommand{\tr}{\text{tr}}
\renewcommand{\a}{\alpha}
\renewcommand{\s}{\sigma}
\newcommand{\us}{{\overline{\sigma}}}
\renewcommand{\d}{\dagger}
\newcommand{\nF}{n_\text{F}}

\begin{document}

\maketitle

\section{Slave Boson}

Aqui $k = \k$ significa todos os números quânticos além do spin.

Lembremos
$$
H = H_{\text{bath}} + H_{\text{hyb}} + H_{\text{imp}},
$$
onde
$$
H_{\text{bath}} = \sum_{k \s} \eps_{k \s} c_{k \s}^\d c_{k\s},
$$
$$
H_{\text{hyb}} = \sum_{k \s}
\Big(
V_{k\s} c_{k\s}^\d f_\s + V_{k\s}^* f_\s^\d c_{k\s}
\Big),
$$
$$
H_{\text{imp}} = U n_{\up} n_\dn + \sum_{k \s} \eps_{0} f_{\s}^\d f_{\s}.
$$

Primeiramente, observemos que o espaço de Hilbert em consideração é o produto tensorial $\H = \H_{\text{imp}} \otimes \H_{\text{conduction}}$, onde os estados de impureza são representados por $\ket{\alpha}$ e os estados de elétrons de condução por $\ket{c}$. Note então que $H_{\text{imp}}$ é diagonalizado pelos estados $\ket{\alpha} = \0$, $\kup$, $\kdn$ e $\2$:
$$
\begin{cases}
\; H_{\text{imp}} \0 = 0 \cdot \0, \\
\; H_{\text{imp}} \kup = \eps_0 \kup, \\
\; H_{\text{imp}} \kdn = \eps_0 \kdn, \\
\; H_{\text{imp}} \2   = (2\eps_0 + U) \2.
\end{cases}
$$
Escrevemos então
$$
H_{\text{imp}} = \sum_\a \eps_\a \ketbra{\a}{\a}.
$$
Temos os elementos de matrizes (todos os outros são nulos)
$$
\mel{\up}{f_\up^\d}{0} = 1, \quad \mel{2}{f_\up^\d}{\dn} = 1,
$$
$$
\mel{\dn}{f_\dn^\d}{0} = 1, \quad \boxed{\mel{2}{f_\dn^\d}{\up} = -1.} \; ?
$$
Portanto, vale que
$$
f_\up^\d = \ketbra{\up}{0} + \ketbra{2}{\dn} = X_{\up, 0} + X_{2, \dn},
$$
$$
f_\dn^\d = \ketbra{\dn}{0} - \ketbra{2}{\up} = X_{\dn, 0} - X_{2, \up},
$$
onde definimos os operadores de Hubbard $X_{p, q} = \ketbra{p}{q}$. Como os estados $\ket{\a}$ são ortonormais, temos a seguinte álgebra de Lie para os operadores de Hubbard:
$$
[X_{p, q}, X_{q', p'}] = \ket{p}\braket{q}{q'}\bra{p'} - \ket{q'}\braket{p'}{p}\bra{q}
= \delta_{q, q'} X_{p, p'} - \delta_{p, p'} X_{q', q}.
$$
Agora utilizamos a seguinte representação para os operadores de Hubbard:

\n

Informalmente:
$$
\begin{cases}
\; \0 \to b, \quad \text{(bosônico)} \\
\; \kup \to s_\up, \quad \text{(fermiônico)} \\
\; \kdn \to s_\dn, \quad \text{(fermiônico)} \\
\; \2 \to a. \quad \text{(bosônico)}
\end{cases}
$$

Formalmente:
$$
\begin{cases}
\; X_{0,0} = b^\d b, \quad \;\; X_{0, \up} = b^\d s_\up,
 \quad X_{0, \dn} = b^\d s_\dn, \quad X_{0, 2} = b^\d a, \\
\; X_{\up,\up} = s_\up^\d s_\up, \quad X_{\up, \dn} = s_\up^\d s_\dn,
 \quad X_{\up, 0} = s_\up^\d b, \quad \; X_{\up, 2} = s_\up^\d a, \\
\; X_{\dn,\dn} = s_\dn^\d s_\dn, \quad X_{\dn, \up} = s_\dn^\d s_\up,
 \quad X_{\dn, 0} = s_\dn^\d b, \quad \; X_{\dn, 2} = s_\dn^\d a, \\
\; X_{2,2} = a^\d a, \quad \; X_{2, \up} = a^\d s_\up,
 \quad X_{2, \dn} = a^\d s_\dn, \quad X_{2, 0} = a^\d b,
\end{cases}
$$
onde impomos os seguintes vínculos:
\begin{itemize}
\item $s_\up$, $s_\dn$ são fermiônicos: $\{s_\up^\d, s_\up\} = \{s_\dn^\d, s_\dn\} = 1$ e $\{s_\up, s_\dn\} = \{s_\up^\d, s_\dn\} = \{s_\up^\d, s_\dn^\d\} = 0$.
\item $a$ e $b$ são bosônicos: $\{a,a^\d\} =1$ e $\{b, b^\d\} = 1$.
\item $a$, $b$ e $s_{\up\dn}$ representam partículas diferentes, então eles também comutam entre si: $[a, b] = [a, s_{\up\dn}] = [b, s_{\up\dn}] = 0$ (o mesmo para os dagger's).
\item $Q \equiv a^\d a + b^\d b + s_\up^\d s_\up + s_\dn^\d s_\dn = 1$ (só existem essas 4 possibilidades).
\end{itemize}

A partir da representação acima para os operadores de Hubbard, utilizando as regras de comutação listadas é possível verificar que $X_{p,q}$ continua satisfazendo a mesma relação algébrica da álgebra de Lie. Esse é o fato que faz a representação pelos operadores acima ser válida.

Devido ao vínculo $Q = 1$ ($Q$ é um projetor), adicionamos o fator $\lambda (Q - 1)$ na hamiltoniana, onde enxergamos $\lambda = -\mu$ como um potencial químico negativo e faremos as contas no ensemble grande canônico ($G$). Para voltarmos ao ensemble canônico, temos o ``truque de Abrikosov''
\begin{equation} \label{eq:abrikosov}
\ev{A} = \lim_{\lambda \to \infty} \frac{\ev{AQ}_G(\lambda)}{\ev{Q}_G(\lambda)}.
\end{equation}
Note que o operador $AQ$ é o operador $A$ restrito ao subespaço das impurezas (onde $Q = 1$). Temos (ignorando extras de Baker-Campbell-Hausdorff)
$$
\ev{QA}_G \stackrel{\lambda \to \infty}{\approx}
\frac{1}{Z_C} \sum_{Q = 1} \mel{n}{Ae^{-\beta H}}{n} +
\sum_{Q = 0} \mel{n}{\cancelto{0}{Q}A\rho_G}{n},
$$
e tomando $A = 1$
$$
\ev{Q}_G \stackrel{\lambda \to \infty}{\approx}
\frac{1}{Z_C} \sum_{Q = 1} \mel{n}{e^{-\beta H}}{n} = \tr(\rho) = 1,
$$
o que nos dá a equação \ref{eq:abrikosov}.

Caso $A$ seja um operador de impureza próprio, então $AQ = A$ e
\begin{equation} \label{eq:abrikosov-imp}
\ev{A} = \lim_{\lambda \to \infty} \frac{\ev{A}_G(\lambda)}{\ev{Q}_G(\lambda)}.
\end{equation}


\section{NCA}

Derivemos a expressão para o NCA nos baseando na seção 7.2 do Hewson (perturbação Keiter-Kimball).


\subsection{Diagramas}

$$
Z = \tr_{f} \tr_{c}\qty{e^{-\beta H}} =
\frac{1}{2\pi i}\int_\Gamma e^{-\beta z} \, \tr_{f} \tr_{c} (z-H)^{-1} \dd{z}.
$$

$$
R_\alpha = \frac{1}{Z_c} \sum_c e^{-\beta E_c} (z-E_\alpha)^{-1} \bra{\alpha}\bra{c}
\sum_{n=0}^\infty \Big[ H_I (z+E_c - H_0)^{-1} \Big]^n \ket{c} \ket{\alpha}.
$$

Para qualquer $\alpha$ temos
$$
R_\alpha^{(0)} = \frac{1}{z - E_\alpha},
$$
ou seja, o propagador livre.

O $H_I$ é dado por
$$
H_I = \sum_{k \s}
\Big(
V_{k\s} s_{\s}^\d b c_{k\s} +
V_{k\s}^* c_{k\s}^\d b^\d s_{\s}
\Big) +
\sum_{k \s}
\Big(
V_{k\s} a^\d s_{\us} c_{k\s} +
V_{k\s}^* c_{k\s}^\d s_{\us}^\d a
\Big).
$$


Temos que o termo com $(H_I)^1$ é nulo para qualquer $\alpha$. Calculemos somente o termo de ordem 2:
$$
R_\alpha^{(2)} = \frac{1}{Z_c} \sum_c e^{-\beta E_c} (z-E_\alpha)^{-1} \bra{\alpha}\bra{c}
H_I (z+E_c - H_0)^{-1} H_I (z+E_c - H_0)^{-1} \ket{c} \ket{\alpha}.
$$
$$
R_\alpha^{(2)} = \frac{1}{Z_c} \sum_c e^{-\beta E_c} (z-E_\alpha)^{-2} \bra{\alpha}\bra{c}
H_I (z+E_c - H_0)^{-1} H_I \ket{c} \ket{\alpha}.
$$
Definamos o termo $\ket{T_{c,\alpha}} = H_I \ket{c} \ket{\alpha}$:
$$
\ket{T_{c,0}} = H_I \ket{c}\ket{0} =
\sum_{k\s} V_{k\s} \Big(
s_{\s}^\d b + a^\d s_{\us}
\Big) \ket{0_c} \ket{0} =
\sum_{k} \Big(
V_{k\up} \ket{\up} + V_{k\dn} \ket{\dn}
\Big).
$$
$$
\ket{T_{c,0}} = \sum_{k\s} V_{k\s} \ket{\sigma}.
$$
De maneira que
$$
R_0^{(2)} = \frac{1}{Z_c} \sum_c e^{-\beta E_c} (z-E_0)^{-2}
\sum_{k\s} \sum_{q\tau} V_{k\s}^* V_{q\tau}
\bra{\sigma} \bra{0_c}
(z+E_c - H_0)^{-1}
\ket{0_c}\ket{\tau}.
$$
$$
R_0^{(2)} = \frac{1}{Z_c} \sum_c e^{-\beta E_c} (z-E_0)^{-2}
\sum_{k\s} \sum_{q\tau} V_{k\s}^* V_{q\tau}
(z + \eps_k - E_\tau)^{-1}
\bra{\sigma} \bra{0_c}
\ket{0_c}\ket{\tau}.
$$
$$
R_0^{(2)} = (z-E_0)^{-1}
\sum_{k\s} V_{k\s}^* V_{k\tau}
(z + \eps_k - E_\tau)^{-1}
\delta_{\s \tau} (z-E_0)^{-1}
\boxed{\sum_c \frac{e^{-\beta E_c }\braket{0_c}{0_c}}{Z_c}}
$$
$$
R_0^{(2)} = (z-E_0)^{-1}
\qty[
\sum_{k\s} \abs{V_{k\s}}^2 \frac{\nF(\eps_{k\s})}{(z + \eps_k - E_\s)^{-1}}
]
(z - E_0)^{-1}.
$$
Corrigindo o polo $E_\s \to E_\s + \Sigma_\s$, conseguimos identificar a self-energy como
$$
\Sigma_0(\omega) =
\sum_{k\s} \abs{V_{k\s}}^2 \frac{\nF(\eps_{k\s})}{(\omega + \eps_k - E_\s)^{-1}} =
\int_{-\infty}^\infty \frac{\dd{\eps}}{\pi} \,
\nF(\eps) \sum_\s \Delta_\s(\eps) G_\s(\omega + \eps),
$$
onde $\Delta(\eps) = \pi \sum_k \abs{V_{k\s}}^2 \delta(\eps - \eps_k)$.

\n

%Repetimos esse cálculo para $\alpha = \; \up$, $\dn$ e $2$.
%$$
%\ket{T_{c,\up}} = H_I \ket{c}\ket{\up} =
%\sum_{k\s} V_{k\s} \Big(
%s_{\s}^\d b + a^\d s_{\us}
%\Big) \ket{0_c} \ket{\up} =
%\sum_{k} \Big(
%V_{k\up} \ket{\up} + V_{k\dn} \ket{\dn}
%\Big).
%$$
%$$
%\ket{T_{c,0}} = \sum_{k\s} V_{k\s} \ket{\sigma}.
%$$

\section{Actual green function}

Lembrando que
$$
f_\s^\d = s_\s^\d b + \s a^\d s_\us \e f_\s = b^\d s_\s + \s s_\us^\d a,
$$
temos
$$
G(t) = -i \theta(t) \zub{f_\s(t)}{f_\s^\d(0)}
$$
e então faremos as contas com calma.


\end{document}
