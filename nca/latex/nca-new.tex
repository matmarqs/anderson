\documentclass[a4paper,fleqn,12pt]{article}
\usepackage[T1]{fontenc}
%\usepackage[utf8]{inputenc}
\usepackage[brazilian]{babel}
\usepackage[left=2.5cm,right=2.5cm,top=3cm,bottom=2.5cm]{geometry}
\usepackage{mathtools}
%\usepackage{amsthm}
\usepackage{amsmath}
\usepackage{nccmath}
\usepackage{amssymb}
\usepackage{amsfonts}
\usepackage{physics}
\usepackage{dsfont}
%\usepackage{mathrsfs}

\usepackage{titling}
\usepackage{indentfirst}

\usepackage{bm}
\usepackage[dvipsnames]{xcolor}
\usepackage{cancel}

\usepackage{xurl}
\usepackage[colorlinks=true]{hyperref}

\usepackage{float}
\usepackage{graphicx}
\usepackage{tikz}
\usepackage{caption}
\usepackage{subcaption}

%%%%%%%%%%%%%%%%%%%%%%%%%%%%%%%%%%%%%%%%%%%%%%%%%%%

\newcommand{\eps}{\epsilon}
\newcommand{\vphi}{\varphi}
\newcommand{\cte}{\text{cte}}

\newcommand{\N}{\mathbb{N}}
\newcommand{\Z}{\mathbb{Z}}
\newcommand{\Q}{\mathbb{Q}}
\newcommand{\R}{\mathbb{R}}
\newcommand{\C}{\mathbb{C}}
\renewcommand{\H}{\hat{H}}
\newcommand{\intR}{\int_{-\infty}^{\infty}}

\newcommand{\0}{\vb{0}}
\newcommand{\1}{\mathds{1}}
\newcommand{\E}{\vb{E}}
\newcommand{\B}{\vb{B}}
\renewcommand{\v}{\vb{v}}
\renewcommand{\r}{\vb{r}}
\renewcommand{\k}{\vb{k}}
\newcommand{\p}{\vb{p}}
\newcommand{\q}{\vb{q}}
\newcommand{\F}{\vb{F}}

\renewcommand{\a}{\hat{a}}
\renewcommand{\b}{\hat{b}}
\renewcommand{\c}{\hat{c}}
\newcommand{\nn}{\hat{n}}

\newcommand{\gf}[2]{\ev{\ev{#1 : #2}}}
\newcommand{\zub}[2]{\ev{\comm{#1}{#2}_\mp}}

\newcommand{\s}[1]{\mathcal{#1}}
%\newcommand{\prodint}[2]{\left\langle #1 , #2 \right\rangle}
\newcommand{\cc}[1]{\overline{#1}}
\newcommand{\Eval}[3]{\eval{\left( #1 \right)}_{#2}^{#3}}

\newcommand{\unit}[1]{\; \mathrm{#1}}

\newcommand{\n}{\medskip}
\newcommand{\e}{\quad \mathrm{e} \quad}
\newcommand{\ou}{\quad \mathrm{ou} \quad}
\newcommand{\virg}{\, , \;}
\newcommand{\ptodo}{\forall \,}
\renewcommand{\implies}{\; \Rightarrow \;}
%\newcommand{\eqname}[1]{\tag*{#1}} % Tag equation with name

\setlength{\droptitle}{-8em}


\title{\Huge{\textbf{NCA}}}
\author{Mateus Marques}

\renewcommand{\tr}{\text{tr}}
\renewcommand{\a}{\alpha}
\renewcommand{\s}{\sigma}
\newcommand{\us}{\overline{\sigma}}
\renewcommand{\d}{\dagger}
\newcommand{\up}{\uparrow}
\newcommand{\dn}{\downarrow}
\newcommand{\nF}{n_\text{F}}

\begin{document}

\maketitle

\section{Diagramas}

$$
Z = \tr_{f} \tr_{c}\qty{e^{-\beta H}} =
\frac{1}{2\pi i}\int_\Gamma e^{-\beta z} \, \tr_{f} \tr_{c} (z-H)^{-1} \dd{z}.
$$

$$
R_\alpha = \frac{1}{Z_c} \sum_c e^{-\beta E_c} (z-E_\alpha)^{-1} \bra{\alpha}\bra{c}
\sum_{n=0}^\infty \Big[ H_I (z+E_c - H_0)^{-1} \Big]^n \ket{c} \ket{\alpha}.
$$

Para qualquer $\alpha$ temos
$$
R_\alpha^{(0)} = \frac{1}{z - E_\alpha},
$$
ou seja, o propagador livre.

O $H_I$ é dado por
$$
H_I = \sum_{k \s}
\Big(
V_{k\s} s_{\s}^\d b c_{k\s} +
V_{k\s}^* c_{k\s}^\d b^\d s_{\s}
\Big) +
\sum_{k \s}
\Big(
V_{k\s} a^\d s_{\us} c_{k\s} +
V_{k\s}^* c_{k\s}^\d s_{\us}^\d a
\Big).
$$


Temos que o termo com $(H_I)^1$ é nulo para qualquer $\alpha$. Calculemos somente o termo de ordem 2:
$$
R_\alpha^{(2)} = \frac{1}{Z_c} \sum_c e^{-\beta E_c} (z-E_\alpha)^{-1} \bra{\alpha}\bra{c}
H_I (z+E_c - H_0)^{-1} H_I (z+E_c - H_0)^{-1} \ket{c} \ket{\alpha}.
$$
$$
R_\alpha^{(2)} = \frac{1}{Z_c} \sum_c e^{-\beta E_c} (z-E_\alpha)^{-2} \bra{\alpha}\bra{c}
H_I (z+E_c - H_0)^{-1} H_I \ket{c} \ket{\alpha}.
$$
Definamos o termo $\ket{T_{c,\alpha}} = H_I \ket{c} \ket{\alpha}$:
$$
\ket{T_{c,0}} = H_I \ket{c}\ket{0} =
\sum_{k\s} V_{k\s} \Big(
s_{\s}^\d b + a^\d s_{\us}
\Big) \ket{0_c} \ket{0} =
\sum_{k} \Big(
V_{k\up} \ket{\up} + V_{k\dn} \ket{\dn}
\Big).
$$
$$
\ket{T_{c,0}} = \sum_{k\s} V_{k\s} \ket{\sigma}.
$$
De maneira que
$$
R_0^{(2)} = \frac{1}{Z_c} \sum_c e^{-\beta E_c} (z-E_0)^{-2}
\sum_{k\s} \sum_{q\tau} V_{k\s}^* V_{q\tau}
\bra{\sigma} \bra{0_c}
(z+E_c - H_0)^{-1}
\ket{0_c}\ket{\tau}.
$$
$$
R_0^{(2)} = \frac{1}{Z_c} \sum_c e^{-\beta E_c} (z-E_0)^{-2}
\sum_{k\s} \sum_{q\tau} V_{k\s}^* V_{q\tau}
(z + \eps_k - E_\tau)^{-1}
\bra{\sigma} \bra{0_c}
\ket{0_c}\ket{\tau}.
$$
$$
R_0^{(2)} = (z-E_0)^{-1}
\sum_{k\s} V_{k\s}^* V_{k\tau}
(z + \eps_k - E_\tau)^{-1}
\delta_{\s \tau} (z-E_0)^{-1}
\boxed{\sum_c \frac{e^{-\beta E_c }\braket{0_c}{0_c}}{Z_c}}
$$
$$
R_0^{(2)} = (z-E_0)^{-1}
\qty[
\sum_{k\s} \abs{V_{k\s}}^2 \frac{\nF(\eps_{k\s})}{(z + \eps_k - E_\s)^{-1}}
]
(z - E_0)^{-1}.
$$
Corrigindo o polo $E_\s \to E_\s + \Sigma_\s$, conseguimos identificar a self-energy como
$$
\Sigma_0(\omega) =
\sum_{k\s} \abs{V_{k\s}}^2 \frac{\nF(\eps_{k\s})}{(\omega + \eps_k - E_\s)^{-1}} =
\int_{-\infty}^\infty \frac{\dd{\eps}}{\pi} \,
\nF(\eps) \sum_\s \Delta_\s(\eps) G_\s(\omega + \eps),
$$
onde $\Delta(\eps) = \pi \sum_k \abs{V_{k\s}}^2 \delta(\eps - \eps_k)$.

\n

Repetimos esse cálculo para $\alpha = \; \up$, $\dn$ e $2$.
$$
\ket{T_{c,\up}} = H_I \ket{c}\ket{\up} =
\sum_{k\s} V_{k\s} \Big(
s_{\s}^\d b + a^\d s_{\us}
\Big) \ket{0_c} \ket{\up} =
\sum_{k} \Big(
V_{k\up} \ket{\up} + V_{k\dn} \ket{\dn}
\Big).
$$
$$
\ket{T_{c,0}} = \sum_{k\s} V_{k\s} \ket{\sigma}.
$$



\end{document}
